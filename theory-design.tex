%!TEX encoding = UTF-8
\documentclass{jsreport}

\usepackage[at]{easylist}
\usepackage[dvipdfmx]{graphicx}
\usepackage{comment}
\usepackage[utf8]{inputenc}
\usepackage[T1]{fontenc}
\usepackage{newpxtext, newpxmath}
\usepackage{lscape}
\usepackage{longtable}
\usepackage[a4paper]{geometry}
\setcounter{secnumdepth}{4}
\setcounter{tocdepth}{4}


% for source code
\usepackage{listings,jlisting}
\lstset{%
  language={C},
  basicstyle={\small},%
  identifierstyle={\small},%
  commentstyle={\small\itshape},%
  keywordstyle={\small\bfseries},%
  ndkeywordstyle={\small},%
  stringstyle={\small\ttfamily},
  frame={tb},
  breaklines=true,
  columns=[l]{fullflexible},%
  numbers=left,%
  xrightmargin=0zw,%
  xleftmargin=3zw,%
  numberstyle={\scriptsize},%
  stepnumber=1,
  numbersep=1zw,%
  lineskip=-0.5ex%
}

% 図->figure 変更
\renewcommand{\figurename}{Fig }

% ソースコード -> SrouceCode
\renewcommand{\lstlistingname}{Source Code}

% 表 -> Table
\renewcommand{\tablename}{Table}

\begin{document}
	\chapter{動作原理・設計}

	\section{ニキシー管駆動回路}
	
	本節では,ニキシー管の駆動回路設計について述べる.

	\subsection{アノード駆動回路}
	ニキシー管はアノードがコモンとなっており,ダイナミック点灯する際はアノードをON/OFFすることで,どの桁を点灯するかを制御する必要ある.

	まずトランジスタで単純に駆動する回路を考えてみる.

	\begin{figure}[htbp]
		\begin{center}
			\includegraphics{image/th2.png}
			\caption{トランジスタを利用した アノードドライブ回路}
			\label{nixie_anode_drive}
		\end{center}
	\end{figure}	

	ただニキシー管は170V程度の高電圧を加えて制御するため,\figurename \ref{nixie_anode_drive} のような回路では,ゲートを駆動する電圧を高くする必要が生じてしまう.
	そのため3.3Vや5V程度の電圧では駆動することが出来ない.

	\begin{figure}[htbp]
		\begin{center}
			\includegraphics{image/th1.png}
			\caption{トランジスタを2個利用した アノードドライブ回路}
			\label{nixie_anode_transistor}
		\end{center}
	\end{figure}

	そこで,耐圧の高いトランジスタやFETを使ったような \figurename \ref{nixie_anode_transistor} 回路がある.
	CTRL端子にトランジスタの閾値電圧以上の電圧をかけるとQ1が駆動し,R1,R2に電流が流れ,これによってQ2の駆動電圧を確保するようになっている.
	しかし,アノードの数だけ耐圧の高いトランジスタ/FETを使うため,値段も高くなる.
	また抵抗等の部品点数アノードの数だけ多くなり実装コストが大きくなる.そのため,トランジスタ/FET以外を使った制御方法を考える必要がある.
	調査したところ,250V程度耐圧を持ったシンク出力が可能な入手性の高いICは存在しない.そのため各アノードはフォトカプラで駆動するのが一番良いと考えられる.
	インターネット上の多くの作例で使われていたTOSHIBAのGaAs赤外LEDとダーリントン型フォトトランジスタを光結合させたDIP型のフォトカプラTLP627は
	ダーリントン型のためスイッチングが遅く,ニキシー管の明るさを確保しきれていない.
	確保するためには,ダーリントン型ではないフォトカプラが必要となる.
	
	20倍程度高速に動作するTL188というTOSHIBA製の表面実装フォトカプラがありそれを利用してドライブすることにする.
	最終的なアノードのドライブ回路を \figurename \ref{nixie_anode_photoscoupler} に示す.

	\begin{figure}[htbp]
		\begin{center}
			\includegraphics{image/th3.png}
			\caption{フォトカプラを利用した アノードドライブ回路}
			\label{nixie_anode_photoscoupler}
		\end{center}
	\end{figure}

	\clearpage

	\subsection{プリバイアス}
	\label{prebias}

	ニキシー管は,意図しない経路で電流が流れ,ダイナミック点灯をするような場合などでは,その電流がゴーストなどの原因になり大きな問題となる.

	\begin{figure}[htbp]
		\begin{center}
			\includegraphics{image/prebias.png}
			\caption{ニキシー管点灯時の電流の経路}
			\label{nixie_prebias}
		\end{center}
	\end{figure}

	黄色の実線の経路が意図された普通の電流経路である.しかし,破線経路のような点灯していない他の電極(電位的に浮いている)に対しても放電が生じ,
	それがまわりまわって隣のニキシー管の1の数字を光らせるという現象が起こる.
	これがゴーストの原因となり,ニキシー管の表示の視認性の低下などが起きる.
	これは点灯していない電極が電気的に浮いてることが原因なので,点灯していないときに放電が起こらない電圧にクランプすることで防止することが可能である.

	\subsection{カソードの駆動回路}
	\ref{prebias} で述べたように単純にカソードをトランジスタでスイッチするだけでは不十分である.ONしてないときに任意のある電圧に出力を設定できるようなICを選定する必要がある.
	クランプ電圧を高くすれば高くするほど余計な意図しない放電が起こりにくいため,クランプ電圧の高くできるものを選定する.

	Digikeyなどで選定した結果,クランプ電圧が90V程度と比較的高く取ることが可能な,texas instrumentsのSN75468を利用することにした.



	\section{昇圧回路}
	一般的なニキシー管の昇圧回路は,目的に書いたようなチョッパ昇圧やコッククロフト・ウォルトン回路用いたものがある.
	チョッパ昇圧を用いたニキシー管向け昇圧回路を以下に示す.

	\begin{figure}[htbp]
		\begin{center}
			\includegraphics[width=\linewidth]{image/upconverter.png}
			\caption{チョッパ昇圧}
			\label{nixie_chop}
		\end{center}
	\end{figure}

	上のチョッパ昇圧では,200Vを作るのにスイッチングスピードなどを考慮すると12V程度の電圧が必要になってしまう.また大電流を流す必要があり,大きなインダクタと,
	大電流をスイッチする高価なFETが必要になる.
	ニキシー管点灯実験用に試作した,昇圧モジュールを以下に示す.

	\begin{figure}[htbp]
		\begin{center}
			\includegraphics[width=0.7\linewidth]{image/upc.png}
			\caption{試作 チョッパ昇圧モジュール}
			\label{nixie_chop_mm}
		\end{center}
	\end{figure}

	\figurename \ref{nixie_chop_mm}を見てわかるように大きなFETや,FETのゲートをドライブする回路も必要になってくる.
	そのため本作品ではフライバックコンバーターを用いた.

	\begin{figure}[htbp]
		\begin{center}
			\includegraphics[width=\linewidth]{image/flyback_converter.png}
			\caption{フライバックコンバーター}
			\label{nixieflyback}
		\end{center}
	\end{figure}

	\figurename \ref{nixieflyback} に示した回路図では,チョッパ昇圧よりも部品数が増えたように感じるが,どの部品も大電流を扱うわけではないので小さく,安価となっている.

	\section{PCB設計}
	以上の考慮を踏まえてPCBを設計した.
	回路図/PCBデータなどを本節では示す.

	\subsection{回路図}

	\newgeometry{top=0.5cm,bottom=1cm,left=0.5cm,right=0.5cm,includefoot}

	\begin{landscape}

		\topskip0pt
    	\vspace*{\fill}
		\begin{figure}[htb!]
			\begin{center}
				\includegraphics[width=0.93\linewidth]{image/nixie_drive_schematic_1.pdf}
				\caption{ドライブ基板 1/2}
				\label{drive_sch2}
			\end{center}
		\end{figure}
		\vspace*{\fill}

		\newpage

		\topskip0pt
    	\vspace*{\fill}
		\begin{figure}[htb!]
			\begin{center}
				\includegraphics[width=0.93\linewidth]{image/nixie_drive_schematic_2.pdf}
				\caption{ドライブ基板 2/2}
				\label{drive_sch1}
			\end{center}
		\end{figure}
		\vspace*{\fill}

		\newpage

		\topskip0pt
		\vspace*{\fill}
		\begin{figure}[htb!]
			\begin{center}
				\includegraphics[width=0.93\linewidth]{image/nixie_mount_schematic.pdf}
				\caption{ニキシー管 マウント基板}
				\label{mount_sch}
			\end{center}
		\end{figure}
		\vspace*{\fill}
		
	\end{landscape}

	\restoregeometry
	

	\subsection{基板}
	本作品に使用した基板は2枚重ねで利用する.
	1枚目はニキシー管を駆動する高電圧を生成する昇圧回路,ニキシー管駆動回路,サーバーとなるESP32を搭載する.
	2枚目はニキシー管のみを実装し,互いを基板対基板コネクタで接続する.
	ニキシー管は,点灯していると電極のスパッタリングが発生し,管表面に電極の成分が付着する.
	その管に付着した成分によってニキシー管が見えなくなるのが寿命であるが,その寿命が来てもドライブ基板はそのままで,ニキシー管基板のみを変えれば再利用できるように
	2枚重ねで設計している.

	\begin{landscape}
		\topskip0pt
		\vspace*{\fill}
		\begin{figure}[htb!]
			\begin{center}
				\includegraphics[width=\textheight]{image/drive_pad.pdf}
				\caption{ニキシー管 ドライブ基板 PAD/外形}
				\label{drive_pad}
			\end{center}
		\end{figure}
		\vspace*{\fill}

		\newpage

		\topskip0pt
		\vspace*{\fill}
		\begin{figure}[htb!]
			\begin{center}
				\includegraphics[width=\textheight]{image/drive_top.pdf}
				\caption{ニキシー管 ドライブ基板 配線レイヤー/外形}
				\label{drive_tt}
			\end{center}
		\end{figure}
		\vspace*{\fill}

		\newpage

		\topskip0pt
		\vspace*{\fill}
		\begin{figure}[htb!]
			\begin{center}
				\includegraphics[width=\textheight]{image/nix_mount.pdf}
				\caption{ニキシー管 マウント基板 配線レイヤー/外形}
				\label{mount_tt}
			\end{center}
		\end{figure}
		\vspace*{\fill}

	\end{landscape}


	\subsection{部品表}

	\begin{center}
	\begin{longtable}{|l|l|l|l|}
		\caption{TableName}

		\label{bom} \\

		\hline
		部品番号 & 値 & パッケージ & 内容 \\ \hline
		\endfirsthead

		\multicolumn{4}{r}{前ページからの続き} \\ \hline
		部品番号 & 値 & パッケージ & 内容 \\ \hline
		\endhead

		\multicolumn{4}{r}{次ページに続く} \\
		\endfoot

		\multicolumn{4}{r}{以上}
		\endlastfoot

		BATT1 & SMTU2032 & SMTU2032 & CR2032 Holder \\ \hline
		C1 & 10uF & C0603 & CAPACITOR \\ \hline
		C2 & 0.1u & C0603 & CAPACITOR \\ \hline
		C3 & 10uF & C0603 & CAPACITOR \\ \hline
		C4 & 0.1uF & C0603 & CAPACITOR \\ \hline
		C5 & 0.1uF & C0603 & CAPACITOR \\ \hline
		C6 & 100uF & C1210 & CAPACITOR \\ \hline
		C7 & 1000pF  & C0603 & CAPACITOR \\ \hline
		C8 & 1000pF  & C0603 & CAPACITOR \\ \hline
		C9 & 10nF & C0603 & CAPACITOR \\ \hline
		C10 & 1000pF  & C0603 & CAPACITOR \\ \hline
		C11 & 0.1uF & C0603 & CAPACITOR \\ \hline
		C12 & 0.1uF & C0603 & CAPACITOR \\ \hline
		C13 & 0.1uF & C0603 & CAPACITOR \\ \hline
		C14 & 0.1uF & C0603 & CAPACITOR \\ \hline
		C15 & 0.1uF & C0603 & CAPACITOR \\ \hline
		C16 & 0.1uF & C0603 & CAPACITOR \\ \hline
		C17 & 1000pF  & C0603 & CAPACITOR \\ \hline
		C18 & 0.1uF & C0603 & CAPACITOR \\ \hline
		C19 & 0.1uF & C0603 & CAPACITOR \\ \hline
		C20 & 1000pF  & C0603 & CAPACITOR \\ \hline
		C21 & 1000pF  & C0603 & CAPACITOR \\ \hline
		C22 & 0.1uF & C0603 & CAPACITOR \\ \hline
		C23 & 0.1u & C0603 & CAPACITOR \\ \hline
		C24 & 10nF & C0603 & CAPACITOR \\ \hline
		C25 & 10uF & C0603 & CAPACITOR \\ \hline
		C26 & 10uF & C0603 & CAPACITOR \\ \hline
		C27 & 0.1u & C0603 & CAPACITOR \\ \hline
		C28 & 0.1uF & C0603 & CAPACITOR \\ \hline
		C29 & UWT1E221MNL1GS & nichicon capasitor & CAPACITOR \\ \hline
		C30 & 0.1u & C0603 & CAPACITOR \\ \hline
		C31 & 33p & C0603 & CAPACITOR \\ \hline
		C32 & 33p & C0603 & CAPACITOR \\ \hline
		C33 & 0.1uF & C0603 & CAPACITOR \\ \hline
		C34 & 1u & C5750 & CAPACITOR \\ \hline
		C35 & 1u & C5750 & CAPACITOR \\ \hline
		D1 & RFN2L4STE25 & SOD-106 & FAST RECOVERY DIODE \\ \hline
		D3 & MMSZ5267BT1G & SOD123 & ZENER DIODE \\ \hline
		J1 & ZX62R-AB-5P & ZX62R-B-5PA & USB micro b \\ \hline
		J2 & DC\_JACK\_2DC0005D100 & 2DC0005D100 & DC JACk \\ \hline
		J3 & FSM-21045-20 & FSM-21045-20 & 20xConnector \\ \hline
		J4 & PIN HEADER 5P & PIN HEADER 5P & 5xPinHeader \\ \hline
		J5 & ZX62R-AB-5P & ZX62R-B-5PA & USB micro b \\ \hline
		J6 & PSM-210223 & PSM-210223 & 20xConnector \\ \hline
		L1 & LPR6235 & LPR6235 & Couple Inductor \\ \hline
		LED1 & RX & CHIPLED\_0603 & LED \\ \hline
		LED2 & TX & CHIPLED\_0603 & LED \\ \hline
		LED6 & LED1 & CHIPLED\_0603 & LED \\ \hline
		Q1 & IRLML6344TRPBFTR & SOT23-N-MOSFET & N-ch MOSFET \\ \hline
		Q2 & MMSS8050-L & SOT23 & Bipolar Transistor NPN \\ \hline
		Q3 & MMSS8050-L & SOT23 & Bipolar Transistor NPN \\ \hline
		Q4 & IRLML6344TRPBFTR & SOT23-N-MOSFET & N-ch MOSFET \\ \hline
		Q5 & FA-238V & FA238-V & Cristal \\ \hline
		R1 & 1k & R0603 & RESISTOR \\ \hline
		R2 & 1k & R0603 & RESISTOR \\ \hline
		R3 & 10k & R0603 & RESISTOR \\ \hline
		R4 & 470 & R0603 & RESISTOR \\ \hline
		R5 & 10k & R0603 & RESISTOR \\ \hline
		R6 & 470 & R0603 & RESISTOR \\ \hline
		R7 & 10k & R0603 & RESISTOR \\ \hline
		R8 & 10k & R0603 & RESISTOR \\ \hline
		R9 & 4.7k & R0603 & RESISTOR \\ \hline
		R11 & 0.1 & R0603 & RESISTOR \\ \hline
		R12 & 1k & R0603 & RESISTOR \\ \hline
		R13 & 20 & R0603 & RESISTOR \\ \hline
		R14 & 174k & R0603 & RESISTOR \\ \hline
		R19 & 220k & R0603 & RESISTOR \\ \hline
		R20 & 4.7k & R0603 & RESISTOR \\ \hline
		R21 & 4.7k & R0603 & RESISTOR \\ \hline
		R22 & 10k & R3216 & RESISTOR \\ \hline
		R23 & 10k & R3216 & RESISTOR \\ \hline
		R24 & 10k & R3216 & RESISTOR \\ \hline
		R25 & 10k & R3216 & RESISTOR \\ \hline
		R26 & 10k & R3216 & RESISTOR \\ \hline
		R27 & 10k & R3216 & RESISTOR \\ \hline
		R28 & 10k & R3216 & RESISTOR \\ \hline
		R29 & 10k & R3216 & RESISTOR \\ \hline
		R30 & 1k & R0603 & RESISTOR \\ \hline
		R31 & 10k & R0603 & RESISTOR \\ \hline
		R32 & 1k & R0603 & RESISTOR \\ \hline
		R33 & 10k & R0603 & RESISTOR \\ \hline
		R34 & 1k & R0603 & RESISTOR \\ \hline
		R35 & 1k & R0603 & RESISTOR \\ \hline
		R36 & 1k & R0603 & RESISTOR \\ \hline
		R37 & 1k & R0603 & RESISTOR \\ \hline
		R38 & 1k & R0603 & RESISTOR \\ \hline
		R39 & 1k & R0603 & RESISTOR \\ \hline
		R40 & 1k & R0603 & RESISTOR \\ \hline
		R41 & 0 & R0603 & RESISTOR \\ \hline
		R42 & 1k & R0603 & RESISTOR \\ \hline
		R43 & 1k & R0603 & RESISTOR \\ \hline
		R44 & 1k & R0603 & RESISTOR \\ \hline
		R45 & 360k & R0603 & RESISTOR \\ \hline
		R46 & 1k & R0603 & RESISTOR \\ \hline
		R47 & 470 & R0603 & RESISTOR \\ \hline
		R48 & 10k & R0603 & RESISTOR \\ \hline
		R49 & 470 & R0603 & RESISTOR \\ \hline
		R50 & 10k & R0603 & RESISTOR \\ \hline
		R51 & 470 & R0603 & RESISTOR \\ \hline
		R52 & 10k & R0603 & RESISTOR \\ \hline
		R57 & 2.4k & R0603 & RESISTOR \\ \hline
		SP1 & PKLCS1212E4001-R1 & PKLCS1212E4001-R1 & Piezoelectric Speaker \\ \hline
		SW1 & SKRPACE010 & SKRPACE010 & Switch \\ \hline
		SW2 & SKRPACE010 & SKRPACE010 & Switch \\ \hline
		SW3 & SKHHLPA010 & SKHHLPA010 & Switch \\ \hline
		SW4 & SKHHLPA010 & SKHHLPA010 & Switch \\ \hline
		SW5 & SKHHLPA010 & SKHHLPA010 & Switch \\ \hline
		SW6 & SKRPACE010 & SKRPACE010 & Switch \\ \hline
		U2 & ESP-WROOM-32 & ESP-WROOM-32 & ESP32 micro controller \\ \hline
		U3 & LM3478 & SOIC8-LT & DC-DC control IC \\ \hline
		U4 & DS3231M & SO08 & DS3231M Real Time Clock \\ \hline
		U5 & SN75468 & R-PDSO-G16 & 8-ch Sink Driver \\ \hline
		U6 & 74HC595D & SO16 & 8-bit SHIFT REGISTER \\ \hline
		U7 & SN75468 & R-PDSO-G16 & 8-ch Sink Driver \\ \hline
		U8 & MCP4017 & SC-70-MICROCHIP & Digital Potention Meter \\ \hline
		U9 & BME280 & LGA-8-2.5X2.5 & Bosch BME280 \\ \hline
		U10 & TLP188 & 11-4M1S & Photocoupler \\ \hline
		U11 & TLP188 & 11-4M1S & Photocoupler \\ \hline
		U12 & TLP188 & 11-4M1S & Photocoupler \\ \hline
		U13 & TLP188 & 11-4M1S & Photocoupler \\ \hline
		U14 & TLP188 & 11-4M1S & Photocoupler \\ \hline
		U15 & TLP188 & 11-4M1S & Photocoupler \\ \hline
		U16 & TLP188 & 11-4M1S & Photocoupler \\ \hline
		U17 & TLP188 & 11-4M1S & Photocoupler \\ \hline
		U18 & ATMEGA328P-AU & TQFP32 & atmel atmega328p microcontroller \\ \hline
		U19 & 74HC595D & SO16 & 8-bit SHIFT REGISTER \\ \hline
		U20 & ADP3338AKCZ-3.3 & SOT233 & Lowdropout Regulator \\ \hline
		U21 & FT232RL & SSOP28 & USB serial convert IC \\ \hline
		U22 & FT232RL & SSOP28 & USB serial convert IC \\ \hline
		N1 & IN-14 & IN-14 & IN-14 \\ \hline
		N2 & IN-14 & IN-14 & IN-14 \\ \hline
		N3 & IN-14 & IN-14 & IN-14 \\ \hline
		N4 & IN-14 & IN-14 & IN-14 \\ \hline
		N5 & IN-14 & IN-14 & IN-14 \\ \hline
		N6 & IN-14 & IN-14 & IN-14 \\ \hline
		N7 & IN-14 & IN-14 & IN-14 \\ \hline
		N8 & IN-14 & IN-14 & IN-14 \\ \hline
		
		\end{longtable}
	\end{center}

	\section{筐体設計}

	\begin{figure}[htbp]
		\begin{center}
			\includegraphics[width=\linewidth]{image/case_bird_eye.png}
			\caption{筐体の外観}
			\label{case}
		\end{center}
	\end{figure}

	筐体として,\figurename \ref{case} を設計した.
	部品として,上面,底面,側面,正面,後面の5種類の板を用いる.
	それぞれの部品の図面を以下5ページに示す.

	% 用紙を横に
	\begin{landscape}
		\topskip0pt
    	\vspace*{\fill}
		\begin{figure}[htb!]
			\begin{center}
				\includegraphics{image/bottom.pdf}
				\caption{筐体底面}
				\label{bottom}
			\end{center}
		\end{figure}
		\vspace*{\fill}

		\newpage

		\topskip0pt
    	\vspace*{\fill}
		\begin{figure}[htb!]
			\begin{center}
				\includegraphics[angle=270]{image/top.pdf}
				\caption{筐体上面}
				\label{top}
			\end{center}
		\end{figure}
		\vspace*{\fill}

		\newpage

		\topskip0pt
    	\vspace*{\fill}
		\begin{figure}[htb!]
			\begin{center}
				\includegraphics{image/side.pdf}
				\caption{筐体側面}
				\label{side}
			\end{center}
		\end{figure}
		\vspace*{\fill}

		\newpage

		\topskip0pt
    	\vspace*{\fill}
		\begin{figure}[htb!]
			\begin{center}
				\includegraphics{image/front.pdf}
				\caption{筐体前面}
				\label{front}
			\end{center}
		\end{figure}
		\vspace*{\fill}

		\newpage

		\topskip0pt
    	\vspace*{\fill}
		\begin{figure}[htb!]
			\begin{center}
				\includegraphics{image/back.pdf}
				\caption{筐体後面}
				\label{back}
			\end{center}
		\end{figure}
		\vspace*{\fill}
	\end{landscape}

	以上の板を,スリーボンド社の一液無溶剤系湿気硬化型弾性接着剤 クリア\footnote{https://www.monotaro.com/g/00332544/}で,
	\figurename \ref{bottom} ~ \figurename \ref{back}の各部品を接着し,\figurename \ref{case_comp} のように箱を作る.

	\begin{figure}[htbp]
		\begin{center}
			\includegraphics[width=\linewidth]{image/case_comp.png}
			\caption{接着した筐体}
			\label{case_comp}
		\end{center}
	\end{figure}



\end{document}